
%%%%%%%%%%%%%%%%%%%%%%%%%%%%%%%%%%%%%%%%%%%%%%%%%%%%%%%%%
\documentclass[inzynierska]{pwr_wmat_praca_dyplomowa}
\autor{Michał Cerazy}
\tytul{Tytuł pracy dyplomowej} 
\tytulang{Tytuł pracy dyplomowej w języku angielskim}
\opiekun{prof. dr hab. Jan Kowalski}
\kierunekstudiow{Matematyka stosowana}
\kierunekstudiowang{Applied Mathematics}
\specjalnosc{--} 
\specjalnoscang{--} 
\streszczenie{Tutaj piszemy krótkie streszczenie pracy (nie powinno być dłuższe niż 530 znaków).}
\streszczenieang{Tutaj piszemy krótkie streszczenie pracy w języku angielskim (nie powinno być dłuższe niż 530 znaków).}
\slowakluczowe{tutaj podajemy najważniejsze słowa kluczowe (łącznie nie powinny być dłuższe niż 150 znaków).}  
\slowakluczoweang{tutaj podajemy najważniejsze słowa kluczowe w języku angielskim (łącznie nie powinny być dłuższe niż 150 znaków)}
%%%%%%%%%%%%%%%%%%%%%%%%%%%%%%%%%%%%%%%%%%%%%%%%%%%%%%%%%
% Definicje, lematy, twierdzenia, przykłady i wnioski
% Komendy wywołujące twierdzenia, definicje, itd., 
% czyli 'theorem', 'definition', 'corollary', itd., 
% można zmienić wedle uznania.
\theoremstyle{plain}
\newtheorem{theorem}{Twierdzenie}
\numberwithin{theorem}{chapter}
\newtheorem{lemma}[theorem]{Lemat} 
\newtheorem{corollary}[theorem]{Wniosek}
\newtheorem{fact}[theorem]{Fakt}
\theoremstyle{definition}
\numberwithin{theorem}{chapter}
\newtheorem{definition}[theorem]{Definicja} 
\newtheorem{example}[theorem]{Przykład}
\newtheorem{note}[theorem]{Uwaga}
%%%%%%%%%%%%%%%%%%%%%%%%%%%%%%%%%%%%%%%%%%%%%%%%%%%%%%%%%
\begin{document}
\frontmatter
\maketitle
\mainmatter
\tableofcontents
%\listoffigures
%\listoftables

{\backmatter \chapter{Wstęp}}
%We wstępie zapowiadamy, o czym będzie praca. Próbujemy zachęcić czytelnika do dalszej lektury, np. krótko informując, dlaczego wybraliśmy właśnie ten temat i co nas w nim zainteresowało.
W dobie powszechnego dostępu do internetu i rozpowszechnienia kultury masowej wiele !!!(do poprawy)

\chapter{Opis ligi NBA, jak wyglada sezon}
National Basketball Association (NBA) została założona 6 czerwca 1946 roku. Pierwotnie była znana jako Basketball Association of America, a swoją obecną nazwę zyskała w roku 1949, kiedy to wchłonęła rywalizującą National Basketball League. Od 2004 roku w NBA gra 30 zespołów, 29 z USA i 1 z Kanady. Liga podzielona jest na dwie konferencje po 15 drużyn, te natomiast składają się z dywizji po 5 ekip. TU WSTAW OBRAZ Z MAPKĄ!!!!!!!!!!!!!!!!. Nazwy wszystkich drużyn z podziałem na konferencje i dywizje:
\begin{itemize}
	\item Konferencja Wschodnia   / do tabeli?
	\begin{itemize}
		\item Atlantic Division
		\begin{itemize}
			\item Boston Celtics
			\item Brooklyn Nets
			\item New York Knicks
			\item Philadelphia 76ers
			\item Toronto Raptors
		\end{itemize}
		\item Southeast Division
		\begin{itemize}
			\item Atlanta Hawks
			\item Charlotte Hornets
			\item Miami Heat
			\item Orlando Magic
			\item Washington Wizards
		\end{itemize}
		\item Central Division
		\begin{itemize}
			\item Chicago Bulls
			\item Cleveland Cavaliers
			\item Detroit Pistons
			\item Indiana Pacers
			\item Milwaukee Bucks
		\end{itemize}
	\end{itemize}	
	\item Konferencja Zachodnia
	\begin{itemize}
		\item Northwest Division
		\begin{itemize}
			\item Denver Nuggets
			\item Minnesota Timberwolves
			\item Oklahoma City Thunder
			\item Portland Trail Blazers
			\item Utah Jazz
		\end{itemize}
		\item Southwest Division
		\begin{itemize}
			\item Dallas Mavericks
			\item Houston Rockets
			\item Memphis Grizzlies
			\item New Orleans Pelicans
			\item San Antonio Spurs
		\end{itemize}
		\item Pacific Division
		\begin{itemize}
			\item Golden State Warriors
			\item Los Angeles Clippers
			\item Los Angeles Lakers
			\item Phoenix Suns
			\item Sacramento Kings
		\end{itemize}
	\end{itemize}
\end{itemize} 

Sezon w NBA składa się z dwóch części: zasadniczej i następującej po niej pucharowej (playoffs). W sezonie zasadniczym każda drużyna rozgrywa 82 mecze, grając z każdym innym zespołem od 2 do 4 gier. Terminarz wyznaczany jest wedle następujących reguł:
\begin{enumerate}
	\item drużyny z różnych konferencji grają ze sobą 2 spotkania (1 na wyjedzie i 1 na własnym boisku),
	\item drużyny z tej samej dywizji grają ze sobą 4 spotkania (2 na wyjedzie i 2 na własnym boisku),
	\item drużyny z tej samej konferencji oraz różnych dywizji grają ze sobą 3 albo 4 spotkania (przynajmniej po jednym na wyjedzie i własnym boisku).
\end{enumerate}
Mecze koszykówki nie mogą zakończyć się remisem (w razie remisu po regulaminowym czasie gry rozgrywa się dogrywki aż do wyłonienia zwycięzcy).
Po zakończeniu sezonu następuje wspomniana wyżej faza pucharowa; wchodzi do niej po 8 najlepszych zespołów z każdej konferencji (w razie takiej samej ilości zwycięstw dla obu zespołów decydują mecze bezpośrednie pomiędzy nimi). W tej fazie drużyny grają ze sobą maksymalnie 7 meczów, czyli zespół, który pierwszy wygra 4 mecze, przechodzi do następnego etapu. W fazie Playoff jasno zdefiniowane są lokalizacje odgrywania spotkań --- lepszy bilans zwycięstw w sezonie zasadniczym skutkuje przewagą parkietu. Seria spotkań grana jest w formacie 2–2–1–1–1, czyli mecze numer 1, 2, 5 i 7 grane są u lepszej z drużyn. Przy doborze przeciwników w tej fazie bierze się pod uwagę pozycję w tabeli konferencji: drużyna z miejsca pierwszego gra z zespołem o ósmym bilansie w danej konferencji, druga z siódmą, i tak dalej. Zwycięzca serii przechodzi do następnego etapu z czterema drużynami, po którym następują finały konferencji --- najlepsze drużyny ze swoich konferencji spotykają się w finałach NBA. Dla lepszego zrozumienia systemu rozgrywek Playoff ZAMIESZCZONO DRZEWKO PONIŻEJ!!!!!!!!!!!!! 

\chapter{Teoria, matematyka}
Monte Carlo, rozkład jednostajny, bootstrap, boxplot?
test shapiro-wilka, gęstość rozkładu, rozklad normalny, qqplot?
\chapter{Metodologia, algorytmy}
opis modeli, jak zebrano dane

\chapter{Wyniki, porownanie modeli}
modelowanie okresem, sezonem, wagami, długością próbki?
symulacja dla 2018-19?
\chapter{Wnioski}
gęstości symulacji powinny mieć rozkład normalny?

{\backmatter \chapter{Podsumowanie}}
%Podsumowanie w pracach matematycznych nie jest obligatoryjne. Warto jednak na zakończenie krótko napisać, co udało nam się zrobić w pracy, a czasem także o tym, czego nie udało się zrobić.

{\backmatter \chapter{Dodatek}}
tabele z prawdopodobieństwami, terminarz sezonu
%Dodatek w pracach matematycznych również nie jest wymagany. Można w nim przedstawić np. jakiś dłuższy dowód, który z pewnych przyczyn pominęliśmy we właściwej części pracy lub (np. w przypadku prac statystycznych) umieścić dane, które analizowaliśmy.

%%%%%%%%%%%%%%%%%%%%%%%%%%%%%%%%%%%%%%%%%%%%%%%%%%%%%%%%%
% BIBLIOGRAFIA
% W tworzeniu bibliografii najlepiej korzystać z BibTex'a, 
% który jest częścią systemu Tex. W naszym przypadku funkcję 
% przechowalni literatury, do której się odwołujemy, pełni 
% plik bibliografia.bib. Nie musimy ręcznie dodawać nowych 
% pozycji do bibliografii. Możemy wejść np. na stronę 
% https://mathscinet.ams.org/mathscinet/index.html, 
% znaleźć odpowiednią pozycję, wybrać ją, a następnie zmienić 
% 'Select alternative format' na BibTeX, skopiować uzyskany 
% tekst, wkleić do pliku bibliografia.bib i skompilować. 
% Gotowe informacje do pliku bibliografia.bib można znaleźć 
% także na https://arxiv.org - gdy znajdziemy interesującą nas 
% pracę, szukamy 'References & Citations' i klikamy 'NASA ADS', 
% a potem 'Bibtex entry for this abstract' 
% i postępujemy tak jak wcześniej.
%%%%%%%%%%%%%%%%%%%%%%%%%%%%%%%%%%%%%%%%%%%%%%%%%%%%%%%%%
\newpage
% w nawiasie klamrowym wpisujemy nazwę pliku z bibliografią w formacie .bib
\bibliografia{bibliografia} 
\end{document}