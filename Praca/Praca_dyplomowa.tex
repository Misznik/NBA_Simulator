
%%%%%%%%%%%%%%%%%%%%%%%%%%%%%%%%%%%%%%%%%%%%%%%%%%%%%%%%%
\documentclass[inzynierska]{pwr_wmat_praca_dyplomowa}
\autor{Michał Cerazy}
\tytul{Tytuł pracy dyplomowej} 
\tytulang{Tytuł pracy dyplomowej w języku angielskim}
\opiekun{dr hab. inż. Krzysztof Burnecki}
\kierunekstudiow{Matematyka Stosowana}
\kierunekstudiowang{Applied Mathematics}
\specjalnosc{--} 
\specjalnoscang{--} 
\streszczenie{Tutaj piszemy krótkie streszczenie pracy (nie powinno być dłuższe niż 530 znaków).}
\streszczenieang{Tutaj piszemy krótkie streszczenie pracy w języku angielskim (nie powinno być dłuższe niż 530 znaków).}
\slowakluczowe{tutaj podajemy najważniejsze słowa kluczowe (łącznie nie powinny być dłuższe niż 150 znaków).}  
\slowakluczoweang{tutaj podajemy najważniejsze słowa kluczowe w języku angielskim (łącznie nie powinny być dłuższe niż 150 znaków)}
%%%%%%%%%%%%%%%%%%%%%%%%%%%%%%%%%%%%%%%%%%%%%%%%%%%%%%%%%
% Definicje, lematy, twierdzenia, przykłady i wnioski
% Komendy wywołujące twierdzenia, definicje, itd., 
% czyli 'theorem', 'definition', 'corollary', itd., 
% można zmienić wedle uznania.
\theoremstyle{plain}
\newtheorem{theorem}{Twierdzenie}
\numberwithin{theorem}{chapter}
\newtheorem{lemma}[theorem]{Lemat} 
\newtheorem{corollary}[theorem]{Wniosek}
\newtheorem{fact}[theorem]{Fakt}
\theoremstyle{definition}
\numberwithin{theorem}{chapter}
\newtheorem{definition}[theorem]{Definicja} 
\newtheorem{example}[theorem]{Przykład}
\newtheorem{note}[theorem]{Uwaga}
%%%%%%%%%%%%%%%%%%%%%%%%%%%%%%%%%%%%%%%%%%%%%%%%%%%%%%%%%
\begin{document}
\frontmatter
\maketitle
\mainmatter
\tableofcontents
%\listoffigures
%\listoftables

{\backmatter \chapter{Wstęp}}
%We wstępie zapowiadamy, o czym będzie praca. Próbujemy zachęcić czytelnika do dalszej lektury, np. krótko informując, dlaczego wybraliśmy właśnie ten temat i co nas w nim zainteresowało.
Dzięki powszechnemu dostępowi do internetu i rozpowszechnieniu kultury masowej amerykańska liga koszykarska NBA zyskała popularność na całym świecie, przyciągając do siebie najlepszych graczy i masy fanów. Dzięki nieprzewidywalności i złożoności tego sportu podejmowano wiele prób przewidywania wyników rozgrywek, które często toczyły się inaczej, niż by zakładano (najlepszym tego przykładem może być sezon 2003/2004, kiedy to nisko notowani Detroit Pistons pokonali faworytów w postaci Los Angeles Lakers). Najlepsi analitycy sportowi starają się ?analizować? każdy aspekt gry i jego wpływ na sytuację na boisku, lecz nikt do tej pory nie był w stanie zaproponować skutecznego modelu opisującego przebieg rozgrywek. 
W niniejszej pracy inżynierskiej podjęto próbę przewidzenia rezultatów wybranego sezonu ligi NBA przy pomocy informacji o wynikach poszczególnych drużyn w poprzednich sezonach. 

\chapter{Opis ligi NBA, jak wyglada sezon}
National Basketball Association (NBA) została założona 6 czerwca 1946 roku. Pierwotnie była znana jako Basketball Association of Americai składała się z 11 zespołów, a swoją obecną nazwę zyskała w roku 1949, kiedy to wchłonęła rywalizującą National Basketball League. Od 2004 roku w NBA gra 30 zespołów, 29 z USA i 1 z Kanady. Liga podzielona jest na dwie konferencje po 15 drużyn, te natomiast składają się z dywizji po 5 ekip. TU WSTAW OBRAZ Z MAPKĄ!!!!!!!!!!!!!!!!. Nazwy wszystkich drużyn z podziałem na konferencje i dywizje:
\begin{table}[]
\begin{tabular}{ |p{5cm}|p{5cm}|p{5cm}|  }
	\hline
	\multicolumn{3}{|c|}{Konferencja Wschodnia} \\
	\hline
	Atlantic Division& Southeast Division&Central Division\\
	\hline
	Boston Celtics& Atlanta Hawks &Chicago Bulls\\\hline
	Brooklyn Nets&Charlotte Hornets&Cleveland Cavaliers\\\hline
	New York Knicks &Miami Heat &Detroit Pistons\\\hline
	Philadelphia 76ers&Orlando Magic &Indiana Pacers\\\hline
	Toronto Raptors&Washington Wizards&Milwaukee Bucks\\\hline
	\hline
\end{tabular}
\end{table}
\begin{table}[]
\begin{tabular}{ |p{5cm}|p{5cm}|p{5cm}|  }
	\hline
	\multicolumn{3}{|c|}{Konferencja Zachodnia} \\
	\hline
	Northwest Division& Southwest Division&Pacific Division\\
	\hline
	Denver Nuggets& Dallas Mavericks &Golden State Warriors\\\hline
	Minnesota Timberwolves&Houston Rockets &Los Angeles Clippers\\\hline
	Oklahoma City Thunder&Memphis Grizzlies &Los Angeles Lakers\\\hline
	Portland Trail Blazers&New Orleans Pelicans &Phoenix Suns\\\hline
	Utah Jazztors&San Antonio Spurs&Sacramento Kings\\
	\hline
\end{tabular}
\end{table}
Dodatkowo, od 2004 roku niektóre kluby zmieniły swoje nazwy lub lokalizacje. W niektórych historycznych zestawieniach lub zbiorach danych mogą widnieć jako (podane w formacie nazwa obecna --- poprzednia):
\begin{itemize}
	\item Charlotte Hornets --- Charlotte Bobcats
	\item Brooklyn Nets --- New Jersey Nets
	\item Oklahoma City Thunder --- Seattle SuperSonics
	\item New Orleans Pelicans --- New Orleans Hornets
\end{itemize}

Sezon w NBA składa się z dwóch części: zasadniczej i następującej po niej pucharowej (playoffs). W sezonie zasadniczym każda drużyna rozgrywa 82 mecze, grając z każdym innym zespołem od 2 do 4 gier. Terminarz wyznaczany jest wedle następujących reguł:
\begin{enumerate}
	\item drużyny z różnych konferencji grają ze sobą 2 spotkania (1 na wyjedzie i 1 na własnym boisku),
	\item drużyny z tej samej dywizji grają ze sobą 4 spotkania (2 na wyjedzie i 2 na własnym boisku),
	\item drużyny z tej samej konferencji oraz różnych dywizji grają ze sobą 3 albo 4 spotkania (przynajmniej po jednym na wyjedzie i własnym boisku).
\end{enumerate}
Mecze koszykówki nie mogą zakończyć się remisem (w razie remisu po regulaminowym czasie gry rozgrywa się dogrywki aż do wyłonienia zwycięzcy).
Po zakończeniu sezonu następuje wspomniana wyżej faza pucharowa; wchodzi do niej po 8 najlepszych zespołów z każdej konferencji (w razie takiej samej ilości zwycięstw dla obu zespołów decydują mecze bezpośrednie pomiędzy nimi). W tej fazie drużyny grają ze sobą maksymalnie 7 meczów, czyli zespół, który pierwszy wygra 4 mecze, przechodzi do następnego etapu. W fazie Playoff jasno zdefiniowane są lokalizacje odgrywania spotkań --- lepszy bilans zwycięstw w sezonie zasadniczym skutkuje przewagą parkietu. Seria spotkań grana jest w formacie 2–2–1–1–1, czyli mecze numer 1, 2, 5 i 7 grane są u lepszej z drużyn. Przy doborze przeciwników w tej fazie bierze się pod uwagę pozycję w tabeli konferencji: drużyna z miejsca pierwszego gra z zespołem o ósmym bilansie w danej konferencji, druga z siódmą, i tak dalej. Zwycięzca serii przechodzi do następnego etapu z czterema drużynami, po którym następują finały konferencji --- najlepsze drużyny ze swoich konferencji spotykają się w finałach NBA. Dla lepszego zrozumienia systemu rozgrywek Playoff ZAMIESZCZONO DRZEWKO PONIŻEJ!!!!!!!!!!!!! 
\\
CZY ZAMIEŚCIĆ OPIS GRY? PODSTAWOWE ZASADY ITP?
\\
Wyniki z sezonu regularnego 2015/2015 prezentują się następująco:
\begin{table}[]
\begin{tabular}{|p{6cm}|p{4cm}|}
	\hline
	\textbf{Drużyna}      & \textbf{Ilość zwycięstw} \\ \hline
	Golden State Warriors & 67 \\ \hline
	Atlanta Hawks & 60 \\ \hline
	Houston Rockets & 56 \\ \hline
	Los Angeles Clippers & 56 \\ \hline
	Memphis Grizzlies & 55 \\ \hline
	San Antonio Spurs & 55 \\ \hline
	Cleveland Cavaliers & 53 \\ \hline
	Portland Trail Blazers & 51 \\ \hline	
	Chicago Bulls & 50 \\ \hline	
	Dallas Mavericks & 50 \\ \hline	
	Toronto Raptors & 49 \\ \hline
	Washington Wizards & 46 \\ \hline
	New Orleans Pelicans & 45 \\ \hline
	Oklahoma City Thunder & 45 \\ \hline
	Milwaukee Bucks & 41 \\ \hline
	Boston Celtics & 40 \\ \hline
	Phoenix Suns & 39 \\ \hline
	Brooklyn Nets & 38 \\ \hline
	Indiana Pacers & 38 \\ \hline
	Utah Jazz & 38 \\ \hline
	Miami Heat & 37 \\ \hline
	Charlotte Hornets & 33 \\ \hline
	Detroit Pistons & 32 \\ \hline
	Denver Nuggets & 30\\ \hline
	Sacramento Kings & 29 \\ \hline
	Orlando Magic & 25 \\ \hline
	Los Angeles Lakers & 21 \\ \hline
	Philadelphia 76ers & 18 \\ \hline
	New York Knicks & 17 \\ \hline
	Minnesota Timberwolves & 16 \\ \hline
\end{tabular}	
\end{table}

Playoffs 2015:
\begin{table}[]
	\begin{tabular}{ |p{5cm}|p{5cm}|p{5cm}|  }
		\hline
		\textbf{Zwycięzca}& \textbf{Przegrany} &\textbf{Wynik}\\\hline		
		\multicolumn{3}{|c|}{Eastern Conference First Round} \\\hline
		Atlanta Hawks& Brooklyn Nets &4-2\\\hline
		Chicago Bulls& Milwaukee Bucks&4-2\\\hline
		Cleveland Cavaliers& Boston Celtics &4-0\\\hline
		Washington Wizards& Toronto Raptors&4-0\\\hline
		\multicolumn{3}{|c|}{} \\\hline
		
		\multicolumn{3}{|c|}{Western Conference First Round} \\\hline
		Golden State Warriors &New Orleans Pelicans &4-0\\\hline
		Houston Rockets& Dallas Mavericks&4-1\\\hline
		Los Angeles Clippers& San Antonio Spurs &4-3\\\hline
		Memphis Grizzlies& Portland Trail Blazers&4-1\\\hline
		\multicolumn{3}{|c|}{} \\\hline

		\multicolumn{3}{|c|}{Eastern Conference Semifinals} \\\hline
		Atlanta Hawks& Washington Wizards &4-2\\\hline
		Cleveland Cavaliers& Chicago Bulls&4-2\\\hline
		\multicolumn{3}{|c|}{} \\\hline
		
		\multicolumn{3}{|c|}{Western Conference Semifinals} \\\hline
		Golden State Warriors& Memphis Grizzlies &4-2\\\hline
		Houston Rockets& Los Angeles Clippers&4-3\\\hline
		\multicolumn{3}{|c|}{} \\\hline
		
		\multicolumn{3}{|c|}{Eastern Conference Finals} \\\hline
		Cleveland Cavaliers& Atlanta Hawks&4-0\\\hline
		\multicolumn{3}{|c|}{} \\\hline
		
		\multicolumn{3}{|c|}{Western Conference Finals} \\\hline
		Golden State Warriors& Houston Rockets&4-1\\\hline
		\multicolumn{3}{|c|}{} \\\hline
		
		\multicolumn{3}{|c|}{Finals} \\\hline
		Golden State Warriors& Cleveland Cavaliers &4-2\\\hline
	\end{tabular}
\end{table}

\chapter{Teoria, matematyka}
Monte Carlo, rozkład jednostajny, bootstrap, rozkład, estymacja, metod nieparametryczne, boxplot, dystrybuanta, wartość oczekiwana?\\
CZY BADAMY NORMALNOŚĆ? test shapiro-wilka, gęstość rozkładu, rozklad normalny, qqplot?

\chapter{Metodologia, algorytmy}
Czy znając wyniki zakończonych rozgrywek jesteśmy w stanie przewidzieć rezultaty przyszłych zawodów?
\\
Dane, które będą wykorzystywane do symulacji sezonu zostały zebrane ze strony sportowej basketballreference.com.
\\
Na potrzeby tej pracy zebrano wyniki starć pomiędzy drużynami począwszy od sezonu 2004/2005 aż do 2017/2018. Początkowo symulowano rozgrywki w sezonie 2014/2015 w celu wybrania najlepszego modelu, a następnie skorzystano z niego, aby przewidzieć wyniki rozgrywek we wciąż trwającym sezonie 2018/2019.
Posiadając ilość wygranych jednej drużyny z drugą na przestrzeni lat dokonano następujących transformacji danych:
\\
W zależności od interwału czasowego, jaki będziemy rozpatrywać, zebrano wyniki w określonych rozgrywkach (na przykład, przy wyznaczaniu wyników sezony 2014/2015 i interwale 5 lat, używać będziemy danych z lat 2009 do 2014). Dzięki uzyskanej w ten sposób liczbie wygranych w możemy stosunek zwycięstw do porażek dla wybranych zespołów (przykład: Boston Celtics i Atlanta Hawks grały ze sobą 10 razy, Jastrzębie wygrały zaledwie 4 razy, dlatego też w starciu z Celtami ich stosunek wygranych do przegranych wynosi $0.4$). Po zastosowaniu tej metody dla wszystkich zespołów uzyskano macierz o rozmiarze 30 wierszy i kolumn zawierającą prawdopodobieństwa na wygraną z każdym zespołem w lidze.
\\
Podczas prób symulacji dokonano intuicyjnego założenia, wedle którego największy wpływ na postawę sezonu mają rozgrywki bezpośrednio go poprzedzające. W tym celu dobrano system wag --- z powodu dynamicznych zmian w lidze, najstarsze sezony otrzymują najmniejszą rangę, która stopniowo zwiększa się, im bliżej do zawodów rozpatrywanych w symulacji. Ważona ilość zwycięstw $Z_i$ $i-$tej drużyny $D_i$ z $j-$tą drużyną $D_j$ wynosi
\begin{equation}
	Z_{ij} = \sum_{k=1}^{n} (1+x\cdot k)\cdot R_{ijk}, 
\end{equation}
gdzie $n$ to ilość sezonów, z których zaciągamy dane, $x$ ustalona waga kolejnych rozgrywek, a $R_{ijk}$ to wynik starć drużyny $D_i$ z drużyną $D_j$ w $k-$tym sezonie. Podczas testowania skuteczności modeli dobierano różne wagi w celu znalezienia tego zwracającego najlepsze predykcje.    
\\
Podczas symulacji program przechodzi przez dokładnie określony terminarz rozgrywek --- drużyna gra z przeciwnikiem tyle razy, ile spotkań wyznaczono w rozkładzie. Algorytmy losowania opisano szczegółowo w ROZDZIALE Z MODELAMI.
\\


\section{Model uśredniony}
Pierwszy ze stworzonych modeli polega na obliczeniu ogólnego stosunku zwycięstw do porażek dla każdej drużyny w wybranym okresie --- wszystkie wygrane zespołu zostają podzielone przez łączną liczbę rozegranych spotkań, wynikiem czego jest liczba z przedziału $[0,1]$ określana jako $P_{i}$, gdzie $i$ to $i-$ta drużyna. Algorytm symulowania wyników spotkań między drużynami wygląda następująco:
\begin{enumerate}
	\item wstaw $i=1$
	\begin{enumerate}
		\item wstaw $j=i$
		\item znajdź drużyny $D_i$ i $D_j$
		\item odczytaj średnie ilości zwycięstw $W_i$ i $W_j$ dla drużyn $D_i$ i $D_j$ 
		\item wyznacz prawdopodobieństwo zwycięstwa $W_{ij}$ przez drużynę  $D_i$ równe $W_{ij}=\frac{W_i}{W_i + W_j}$   
		\item w terminarzu znajdź liczbę spotkań $S_{ij}$ pomiędzy drużynami $D_i$ i $D_j$
			\begin{enumerate}
				\item symuluj liczbę $U$ z rozkładu jednostajnego $U\sim U[0,1]$ 
			\item jeżeli $W_{ij} \leq U,$ to zwiększ licznik zwycięstw drużyny $D_i$, w przeciwnym razie zwiększ licznik zwycięstw drużyny $D_j$
			\item powtórz $S_{ij}$ razy
			\end{enumerate}
		\item wstaw $j=j+1$
	\end{enumerate}
	\item wstaw $i=i+1$
	\item jeżeli $i> 30$, to zakończ algorytm
\end{enumerate} 

Przy testowaniu skuteczności tego modelu 

\section{Model rywalizacji}
Drugi z zaproponowanych modeli zakłada zwracanie uwagi na historyczne wyniki przeciwko konkretnej drużynie. W zawodowym sporcie niejednokrotnie można trafić na zażarte rywalizacje między dwoma klubami lub zwykłą łatwość w pokonaniu szczególnego przeciwnika. Algorytm symulowania wyników spotkań między drużynami wygląda następująco:
\begin{enumerate}
	\item wstaw $i=1$
	\begin{enumerate}
		\item wstaw $j=i$
		\item znajdź drużyny $D_i$ i $D_j$
		\item odczytaj z macierzy wyników stosunek zwycięstw $W_{ij}$ drużyny $D_i$ przeciw drużynie $D_j$   
		\item w terminarzu znajdź liczbę spotkań $S_{ij}$ pomiędzy drużynami $D_i$ i $D_j$
		\begin{enumerate}
			\item symuluj liczbę $U$ z rozkładu jednostajnego $U\sim U[0,1]$ 
			\item jeżeli $W_{ij} \leq U,$ to zwiększ licznik zwycięstw drużyny $D_i$, w przeciwnym razie zwiększ licznik zwycięstw drużyny $D_j$
			\item powtórz $S_{ij}$ razy
		\end{enumerate}
		\item wstaw $j=j+1$
	\end{enumerate}
	\item wstaw $i=i+1$
	\item jeżeli $i> 30$, to zakończ algorytm
\end{enumerate} 

\section{Model symulacji fazy pucharowej}
Po symulacji całego sezonu, czyli 1230 spotkań, 8 najlepszych drużyn z każdej konferencji przechodzi do fazy Playoff, gdzie toczy rozgrywki zgodnie z systemem opisanym we WSTĘPIE. Na tym etapie rozgrywek symulacja spotkań różni się od części zasadniczej: zamiast jednego z zasugerowanych wcześniej modeli korzysta się ze wcześniejszej symulacji fazy zasadniczej. W celu oddania trendów panujących w wygenerowanych rozgrywkach (a mianowicie potencjalnych kontuzjach, spadkach lub zwyżkach formy), użyta zostaje jedynie informacja o ilości wygranych przed rozpoczęciem Playoffów. Algorytm symulowania tej fazy jest postaci:
\begin{enumerate}
	\item wybierz drużyny $D_i$ i $D_j$
	\item odczytaj symulowane ilości zwycięstw $W_i$ i $W_j$ dla wybranych drużyn $D_i$ i $D_i$
	\item wyznacz prawdopodobieństwo zwycięstwa $W_{ij}$ przez drużynę  $D_i$ równe $W_{ij}=\frac{W_i}{W_i + W_j}$ 
	\item wstaw liczniki zwycięstw $Z_i=0$ i $Z_j=0$ 
		\begin{enumerate}
			\item symuluj liczbę $U$ z rozkładu jednostajnego $U\sim U[0,1]$ 
			\item jeżeli $W_{ij} \leq U,$ wstaw $Z_i=Z_i+1$, w przeciwnym razie wstaw $Z_j=Z_j+1$
			\item powtarzaj dopóki  $Z_i=4$ lub $Z_j=4$
		\end{enumerate}
	\item jeżeli $Z_i=4$, to przenieś drużynę $D_i$ do następnego etapu, w przeciwnym razie przenieś drużynę $D_j$
\end{enumerate} 

Ilości zwycięstw drużyn w kolejnych symulowanych rozgrywkach są zapisywane i zapamiętywane, podobnie jak informacje o przejściach do kolejnych faz rozgrywek pucharowych. 


\chapter{Wyniki, porownanie modeli}
modelowanie okresem, sezonem, wagami, długością próbki?
symulacja dla 2018-19?
\chapter{Wnioski}
gęstości symulacji powinny mieć rozkład normalny?

{\backmatter \chapter{Podsumowanie}}
%Podsumowanie w pracach matematycznych nie jest obligatoryjne. Warto jednak na zakończenie krótko napisać, co udało nam się zrobić w pracy, a czasem także o tym, czego nie udało się zrobić.

{\backmatter \chapter{Dodatek}}
tabele z prawdopodobieństwami, terminarz sezonu
%Dodatek w pracach matematycznych również nie jest wymagany. Można w nim przedstawić np. jakiś dłuższy dowód, który z pewnych przyczyn pominęliśmy we właściwej części pracy lub (np. w przypadku prac statystycznych) umieścić dane, które analizowaliśmy.

%%%%%%%%%%%%%%%%%%%%%%%%%%%%%%%%%%%%%%%%%%%%%%%%%%%%%%%%%
% BIBLIOGRAFIA
% W tworzeniu bibliografii najlepiej korzystać z BibTex'a, 
% który jest częścią systemu Tex. W naszym przypadku funkcję 
% przechowalni literatury, do której się odwołujemy, pełni 
% plik bibliografia.bib. Nie musimy ręcznie dodawać nowych 
% pozycji do bibliografii. Możemy wejść np. na stronę 
% https://mathscinet.ams.org/mathscinet/index.html, 
% znaleźć odpowiednią pozycję, wybrać ją, a następnie zmienić 
% 'Select alternative format' na BibTeX, skopiować uzyskany 
% tekst, wkleić do pliku bibliografia.bib i skompilować. 
% Gotowe informacje do pliku bibliografia.bib można znaleźć 
% także na https://arxiv.org - gdy znajdziemy interesującą nas 
% pracę, szukamy 'References & Citations' i klikamy 'NASA ADS', 
% a potem 'Bibtex entry for this abstract' 
% i postępujemy tak jak wcześniej.
%%%%%%%%%%%%%%%%%%%%%%%%%%%%%%%%%%%%%%%%%%%%%%%%%%%%%%%%%
\newpage
% w nawiasie klamrowym wpisujemy nazwę pliku z bibliografią w formacie .bib
\bibliografia{bibliografia} 
\end{document}